\section{Conclusioni}
\begin{frame}
	\frametitle{Conclusioni}
	\centering
	
	\begin{itemize}
		\justifying
		\item Il presente studio evidenzia il valore aggiunto del nuovo modello proposto nel campo della modellazione funzionale spazio-temporale, distinguendosi dal modello f-HDGM grazie all'introduzione del parametro di interazione spaziale $\rho$;
		\item una delle limitazioni riguarda il processo di stima del nuovo parametro, poiché la cross validazione richiede risorse computazionali significative;
	\end{itemize}	
\end{frame}

\begin{frame}
	\centering
	\begin{itemize}
		\justifying
		\item é necessaria la modifica dell'algoritmo EM affinché sia in grado di minimizzare anche la funzione $m(\rho)$, ma ciò richiede ulteriori sessioni di ricerca e sviluppo per affrontare questioni ancora aperte, come la verifica dell'identificabilità e l'inizializzazione corretta di $\rho$.		
		\item per migliorare l'analisi sul bike sharing, si potrebbe considerare l'utilizzo di un dataset pluriennale per catturare la componente stagionale del fenomeno e l'adozione di variazioni metodologiche come il clustering dei punti di ritiro per permettere la stima di modelli spazio-temporali locali, riducendo l'assunzione che il parametro $\rho$ sia spazio-invariante.
	\end{itemize}	
\end{frame}