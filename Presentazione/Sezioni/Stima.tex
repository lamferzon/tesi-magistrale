\section[Stima EM del modello fp-HDGM]{Stima EM del modello fp-HDGM}

\begin{frame}
	\frametitle{Obiettivo della stima EM}
	\justifying
	
	Sia:
	\begin{equation*}
		\boldsymbol{\theta} = \left(\boldsymbol{c}_\epsilon^\top, \boldsymbol{c}_\beta^\top, \mathbf{g}^\top, \mathbf{v}^\top, \boldsymbol{\lambda}^\top, \textcolor{red}{\rho}\right)^\top
	\end{equation*}
	il vettore dei parametri. 
	\newline \par L'obiettivo dell'\textbf{algoritmo Expectation-Maximization (EM)} è determinare, mediante un processo iterativo, la stima a massima verosimiglianza $\boldsymbol{\theta}_{MLE}$ \textbf{in presenza di dati mancanti}, in questo caso la componente latente $\mathbf{z}(\mathbf{s}, t)$.
\end{frame}

\begin{frame}
	\frametitle{I passi dell'algoritmo}
	\justifying
	
	Ogni iterazione dell'algoritmo EM consiste in due passi:
	\begin{itemize}
		\justifying
		\item \textbf{passo E}: a partire dalla stima corrente dei parametri $\boldsymbol{\theta}_n$ e dai dati disponibili $\mathbf{y}$, quelli mancanti vengono prima stimati e poi impiegati per determinare il \textit{valore atteso condizionato} $Q(\boldsymbol{\theta},\boldsymbol{\theta}_n)$, una semplificazione della stima corrente della funzione di verosimiglianza $L(\boldsymbol{\theta}|\boldsymbol{\theta}_n)$, ovvero $l(\boldsymbol{\theta}|\boldsymbol{\theta}_n)$;
		\item \textbf{passo M}: $Q(\boldsymbol{\theta},\boldsymbol{\theta}_n)$ viene ottimizzata per determinare $\boldsymbol{\theta}_{n+1}$, assumendo che i dati mancanti siano noti. Le stime di questi, ottenute precedentemente nel passo E, sono utilizzate al posto dei dati mancanti effettivi.
	\end{itemize}
\end{frame}

\begin{frame}
	\frametitle{La funzione di verosimiglianza}
	
	\begin{equation*}
		\begin{split}
			-2\ln L(\boldsymbol{\theta}; Y, Z, X) & =  T\ln|\Sigma_\epsilon\left(\textcolor{black}{\mathbf{c}_\epsilon}\right)| \\
			& + \textcolor{red}{\sum_{t=1}^{T} \left(H^{-1}(\rho)\cdot\mathbf{y}_t - \boldsymbol{\mu}_t(\mathbf{c}_\beta)\right)^\top\Sigma_\epsilon^{-1}\left(H^{-1}(\rho)\cdot\mathbf{y}_t - \boldsymbol{\mu}_t(\mathbf{c}_\beta)\right)} \\
			& + \ln|\Sigma_0| \\
			& + (\mathbf{z}_0 - \boldsymbol{\mu}_0)^\top\Sigma_0^{-1}(\mathbf{z}_0 - \boldsymbol{\mu}_0) \\
			& + T\ln |\Sigma_\eta(\textcolor{black}{\mathbf{v},\boldsymbol{\lambda}})| \\
			& + \sum_{t=1}^{T}\left(\mathbf{z_t}-\tilde{G}(\textcolor{black}{\mathbf{g}})\cdot\mathbf{z}_{t-1}\right)^\top\Sigma_\eta^{-1}(\textcolor{black}{\mathbf{v},\boldsymbol{\lambda}})\left(\mathbf{z_t}-\tilde{G}(\textcolor{black}{\mathbf{g}})\cdot\mathbf{z}_{t-1}\right)
		\end{split}
		\label{eq_fin_verosimiglianza}
	\end{equation*}
	
\end{frame}