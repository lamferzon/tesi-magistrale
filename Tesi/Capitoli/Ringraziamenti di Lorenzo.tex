\addchap{Ringraziamenti di Lorenzo}

Cosa ti piacerebbe fare da grande? Una domanda classica alla quale noi tutti abbiamo dovuto provare a rispondere sin dai tempi delle scuole elementari, un quesito la cui risposta ha subito diversi cambiamenti durante il mio percorso formativo. Sinceramente un’idea ben chiara non l’ho mai avuta; da piccolo sognavo di fare dello sport la mia professione, poi, al termine delle scuole medie, decisi di frequentare il liceo scientifico aeronautico perché all’epoca ero affascinato dal volo e da quello che sarei potuto diventare, ovvero un capitano di Lufthansa oppure, ancora meglio, un pilota militare. Dopodiché, nonostante le innumerevoli ore passate a giocare a Prepar3D, ergo il Flight Simulator con i muscoli, mi resi conto che la via del volo non faceva per me; infatti in quinta liceo, qualche mese prima dell’esame di maturità, ero sul punto di preiscrivermi all’ITS (Istituto Tecnico Superiore) di San Paolo d’Argon per imparare una professione, una decisione che cambiai a luglio, sostenendo il TOLC e immatricolandomi alla facoltà di ingegneria informatica a Dalmine. All’epoca conoscevo, in maniera per di più approssimativa, com’era fatto un computer soltanto dal punto di vista hardware, quindi il dubbio che il mondo dell’informatica facesse per me l’ho sempre avuto. Si può dire che la mia sia stata una decisione poco motivata e presa all’ultimo? Probabilmente sì perché se avessi ascoltato il cuore, sarei un preparatore atletico in questo momento; ciononostante non posso non essere felice e soddisfatto di quello che è stato il mio percorso accademico, a tal punto che indietro non tornerei.
\par La facoltà di ingegneria, grazie alla forma mentis e alla formazione scientifica a tuttotondo che essa impartisce, mi ha reso curioso e mi ha fatto apprezzare quella che io ritengo essere la più bella applicazione della matematica, ossia la statistica e i suoi seguiti, machine learning e intelligenza artificiale di cui tanto di parla negli ultimi anni. Un debole per la matematica l’ho sempre avuto e non esiste facoltà diversa da ingegneria che meglio ti insegna ad applicarla al problem solving. Se dovessi  motivare, adesso, la scelta di andare a Dalmine, allora essa è sicuramente quella che più mi ha condizionato; tuttavia l’ha fatto inconsciamente perché spesse volte ci si rende conto soltanto alla fine di un viaggio, e delle difficoltà che esso ti pone dinanzi, che quello appena percorso era il cammino giusto.
\par Quindi, un primo grazie va alla mia cara università e ai progetti del corso di modelli stocastici; essi infatti, oltre ad avermi fatto capire che la statistica è l’unica materia in grado non solo di prevedere il futuro ma anche di valutare l’incertezza di una predizione, mi ha aiutato a scegliere il percorso DSDE (Data Science Data Engineering) per la mia magistrale. Qui ho incontrato Nicola, il milanista con il quale ho riso e scherzato negli ultimi due anni, oltre ad aver lavorato a questo progetto, ovviamente. Quello che sarebbe dovuto essere un piccolo caso di studio, alla fine è diventato un nuovo modello statistico. Di dubbi sul fatto che saremmo riusciti a scrivere le formule di stima ce n’erano, ciononostante insieme ce l’abbiamo fatta e non posso non ringraziarlo per il suo supporto; per me non è solo un collega, ma anche un amico, un vero amico.
Un grazie va anche al mio relatore, al prof. Francesco Finazzi. Egli mi ha seguito magistralmente durante l’intera stesura della tesi, mostrandosi sempre disponibile a rispondere ai miei quesiti, anche il fine settimana. È stata la prima persona a credere in me e nella buona riuscita di questa tesi, quindi gli sono riconoscente.
\par Poi ci sono i miei colleghi di ufficio; perché mai dovrei citarli? Potrei rispondere dicendo che sono stato obbligato, ma non è vero perché è anche grazie al clima di entusiasmo che ogni giorno mi fanno respirare in ufficio che sono riuscito in primis a ultimare la scrittura di questo elaborato senza incorrere in un esaurimento nervoso, in secundis a effettuare la consegna utilizzando l’hotspot di Mario perché il mio non voleva saperne di funzionare. Mi vogliono bene e mi fanno sentire sempre più parte della squadra, un team sul quale posso sempre contare e io sono fiero di farne parte.
Mi sto forse dimenticando di qualcuno? Ebbene sì, c’è anche Asya, colei senza la quale questi mesi di stesura matta e disperatissima sarebbero stati decisamente più impervi e difficoltosi. Mi è stata sempre vicino, dalla mattina alla sera, strappandomi un sorriso ogniqualvolta i conti non tornavano oppure l’implementazione in MATLAB di un algoritmo non faceva il suo dovere. Le voglio bene e sono fiero di averla accanto in quello che io ritengo sia uno dei giorni più belli della mia vita.
\par Dopodiché, c’è la mia famiglia, in particolare mamma e papà che questo momento l’hanno sempre desiderato. Per loro è un sogno che diventa realtà vedere proprio figlio con la corona da alloro in testa in quanto è la realizzazione dei tanti sacrifici fatti per formarmi e per permettermi di raggiungere i miei obiettivi. Sono fiero dei genitori che sono e spero di averli resi fieri del proprio figlio, del loro bambino che ora più definirsi ufficialmente un dottore in ingegneria.
\par Infine, un pensiero speciale va al mio caro nonno, che da lassù ogni dì mi guarda e mi infonde la forza, tenace com’era, di affrontare anche gli ostacoli più impervi; questo traguardo è anche tuo.