\addchap[Introduzione]{Introduzione}

I modelli spazio-temporali sono un fondamentale strumento concettuale utilizzato in una vasta gamma di discipline scientifiche per comprendere e analizzare fenomeni che si sviluppano nello spazio e nel tempo. Questi sistemi statistici forniscono un framework indispensabile per la descrizione e la predizione di eventi fisici, sociali, economici e naturali, permettendo agli studiosi di esplorare le relazioni tra le varie variabili in contesti complessi.
\par L'importanza di tali modelli nella vita quotidiana è evidente in molteplici contesti. Ad esempio, nel campo delle scienze ambientali, consentono di monitorare e modellare i cambiamenti climatici, le variazioni negli ecosistemi e la diffusione di inquinanti. Nell'ambito della pianificazione urbana, invece, sono utilizzati per ottimizzare la distribuzione delle risorse e dei servizi, migliorando la qualità della vita nelle città. Inoltre, nel settore dei trasporti, contribuiscono a ottimizzare le reti di trasporto e a gestire il traffico in tempo reale, riducendo congestioni e tempi di percorrenza.
\par Uno dei modelli spazio-temporali proposti in letteratura è il \textit{Functional Hidden Dynamic Geostatistical Model}(\textit{f-HDGM})~[\cite{paper_f_HDGM}]. Esso, oltre a essere un modello funzionale, è anche \textit{state-space}; una variabile latente consente di rappresentare e modellare fenomeni non direttamente osservabili ma che influenzano il comportamento del sistema. Si ipotizzi, per esempio, di voler descrivere la concentrazione di anidride carbonica $y_{CO_2}(\mathbf{s}, l, t)$ rilevata nel tempo da una stazione di misura in funzione dell'umidità relativa $x_{rel}(\mathbf{s}, l, t)$. Probabilmente il loro legame cambia a seconda dell'ora del giorno, un effetto che, inoltre, potrebbe mutare da un giorno all'altro; infatti, non è detto che l'umidità relativa descriva la concentrazione dell'inquinante in oggetto sempre nello stesso modo, sia in estate che in inverno, in tutti i punti spaziali. Per modellare questa variabilità spazio-temporale, non direttamente osservabile dai dati, subentra la componente latente.
\par Uno dei limiti del modello f-HDGM è l'impossibilità di tener conto dell'interazione che potrebbe esistere tra punti di misura. Si prenda come esemplificazione il potenziale di mercato spaziale; esso rappresenta il volume di vendite previsto quando un determinato prodotto viene commercializzato in un'area di scambio. Le vendite sono previste essere elevate se un negozio viene aperto in una posizione caratterizzata da un alto potenziale di mercato spaziale, mentre si prevede una loro riduzione ove il potenziale mercato risulta essere basso. Al fine di ottenere una stima riguardo al volume degli scambi commerciali, la valutazione del potenziale di mercato spaziale emerge come un aspetto cruciale. 
\par La reciproca influenza tra i negozi, dove il volume delle vendite di ciascun punto è influenzato dalla presenza degli altri, rende imprescindibile considerare tale interazione al fine di stimare correttamente il potenziale di mercato spaziale. Da tale necessità deriva l'obiettivo di questo lavoro: estendere il modello f-HDGM affinché tenga conto anche di quest'aspetto.
\par Dopo aver introdotto nel primo capitolo il concetto di analisi funzionale e di stima EM, nel secondo e nel terzo si entra in medias res illustrando prima l'estensione del modello e poi le formule di stima dei suoi parametri. Nel quarto capitolo, invece, le nozioni teoriche vengono applicate a un caso di studio riguardante il fenomeno del bike sharing in una metropoli statunitense. Infine, il quinto capitolo chiude il lavoro traendo le considerazioni finali e illustrando i possibili sviluppi futuri. 