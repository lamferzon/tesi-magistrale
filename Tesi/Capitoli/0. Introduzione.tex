\addchap[Introduzione]{Introduzione}

Da sottolineare che attualmente non esiste una versione multi-variata\footnote{$\mathbf{y}(\mathbf{s}, l, t)\in\mathbb{R}^{p\times 1}$, con $p>1$.} dell'f-HDGM.
\par Si ipotizzi, per esempio, di voler descrivere la concentrazione di anidride carbonica $y_{CO_2}(\mathbf{s}, l, t)$ rilevata nel tempo da una stazione di misura in funzione dell'umidità relativa $x_{rel}(\mathbf{s}, l, t)$. Probabilmente il loro legame cambia a seconda dell'ora del giorno, un effetto che, inoltre, potrebbe mutare da un giorno all'altro; infatti, non è detto che l'umidità relativa descriva la concentrazione dell'inquinante in oggetto sempre nello stesso modo, sia in estate che in inverno, in tutti i punti spaziali. Per modellare questa variabilità spazio-temporale subentrano i coefficienti $z(\mathbf{s}, t)$, ovvero la componente latente.
\par Si prenda ora in considerazione come esemplificazione il potenziale di mercato spaziale. Quest'ultimo è la distribuzione spaziale del potenziale di mercato\footnote{rappresenta il volume di vendite previsto quando un determinato prodotto viene commercializzato.} su un'area di scambio.
Le vendite sono previste essere elevate se un negozio viene aperto in una posizione caratterizzata da un alto potenziale di mercato spaziale, mentre si prevede una riduzione in presenza di un basso potenziale di mercato nella collocazione spaziale.
Al fine di ottenere una stima riguardo al volume degli scambi commerciali, la valutazione del potenziale di mercato spaziale emerge come un aspetto cruciale. Nell'ambito di questa ricerca, si ipotizza che il prodotto sia già sul mercato e che siano disponibili i dati di vendita o noleggio relativi ai negozi distribuiti nello spazio. Di conseguenza, l'obiettivo è quello di stimare il potenziale di mercato spaziale attraverso l'analisi dei dati di vendita, delle caratteristiche spaziali dell'area di scambio assumendo che il fenomeno sottostante implichi, anche se in modo marginale, una qualche forma di relazione o connessione tra le stazioni di osservazione. La reciproca influenza tra i negozi, dove il volume delle vendite di ciascun punto è influenzato dalla presenza degli altri, rende imprescindibile considerare tale interazione al fine di stimare correttamente il potenziale di mercato spaziale.
