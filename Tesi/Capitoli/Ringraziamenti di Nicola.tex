\addchap{Ringraziamenti di Nicola}

Il mio percorso accademico è stato un viaggio caratterizzato da sfide e ostacoli, ma anche di piacevoli successi. Io che arrivavo da un istituto tecnico, dove le ore scientifiche erano ridotte all'osso (sopratutto quelle di informatica), non ero così sicuro di farcela. Dopo una prima sessione segnata da un unico esame superato, seguita da un periodo di sei sette fuori dall'università a causa di uno spiacevole imprevisto, ho affrontato il mio percorso con sacrificio e perseveranza fino ad arrivare al tanto desiderato traguardo. Di seguito, desidero esprimere la mia più profonda gratitudine alle persone speciali che hanno contribuito alla mia crescita. Grazie a loro, sono diventato la persona, l'uomo e l'ingegnere che sono oggi.
\par In primis, desidero rivolgere un sentito ringraziamento ai professori dell'Università di Bergamo, riconoscendo e apprezzando la loro inestimabile conoscenza e professionalità. In particolare, al relatore di questa tesi Francesco Finazzi, con cui ho condiviso un forte interesse per la statistica e i modelli spazio-temporali. Un ringraziamento speciale va anche ai miei amici e compagni di viaggio, Matteo e Wasim, che hanno giocato ruoli essenziali nel mio percorso, collaborando con me nei progetti; non posso fare a meno di ringraziare in modo speciale Lorenzo, il cui contributo è stato fondamentale per portare a termine questo prezioso lavoro. La sua pazienza, costanza e positività nel gestire situazioni complesse sono state inestimabili. Rimarranno indelebili nella mia memoria le notti trascorse insieme, immersi nel codice o nelle formule di questa tesi, dopo lunghe ore di lavoro.
\par Voglio ringraziare tutti i membri della mia famiglia per il loro incoraggiamento e supporto incondizionato. In particolare i miei genitori, Alessandra e Massimo, che hanno sempre creduto in me e mi hanno permesso di fare le scelte che volevo. È difficile trovare le parole per descrivere i sacrifici che hanno fatto e che fanno tutti i giorni. Siano certi che resteranno sempre nel mio cuore, anche se so che non potrò mai ripagarli adeguatamente. Ai miei fratelli Matteo e Davide, che hanno sempre sostenuto e compreso le mie scelte. A Silvia, perché la sua lunga e ambiziosa carriera accademica è stata una fonte di ispirazione e guida per me.
Un ringraziamento particolare va alla mia ragazza Giorgia, per il suo amore e sostegno nei momenti difficili. la sua presenza costante e il suo incoraggiamento mi hanno sostenuto nei momenti più bui di questo percorso. Infine, un grazie alla mia fedele compagna a quattro zampe, Tesla, che con le sue passeggiate ha contribuito a schiarirmi le idee nei momenti di tensione.