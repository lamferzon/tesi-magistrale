\chapter[Il concetto di geo-potenziale condizionato e il modello fp-HDGM]{Il concetto di geo-potenziale condizionato e definizione del modello fp-HDGM}

\textcolor{red}{Introduzione al capitolo\dots}

\section[Il geo-potenziale condizionato]{Il geo-potenziale condizionato}
\textcolor{red}{Concetto generale di geo-potenziale condizionato e spiegazione della matrice $H$ e dei suoi termini, ovvero $h_i$ e $\rho$\dots}

\subsection[Differenza tra geo-potenziale puro e condizionato]{Differenza tra geo-potenziale puro e condizionato}
\textcolor{red}{Avvalersi del caso di studio inerente il potenziale di vendita \dots}

\section[Il modello fp-HDGM]{Il modello fp-HDGM}
\textcolor{red}{Finalità del nuovo modello\dots}

\subsection[Equazioni del modello]{Equazioni del modello}
\textcolor{red}{Due equazioni, $y$ e $w$ (output del modello base)\dots}

\subsection[Parametri da stimare]{Parametri da stimare}
\textcolor{red}{Sono quelli del modello f-HDGM e $\rho$\dots}

\paragraph[Differenza tra $\theta$ e $\rho$]{Differenza tra $\theta$ e $\rho$}

\subsection[Simulazione di una mappa di geo-potenziale]{Simulazione di una mappa di geo-potenziale}
\textcolor{red}{Grafico del pontenziale puro ($\rho=0$)\dots}\\
\textcolor{red}{Grafici sotto forma di sotto-figure del potenziale condizionato con diversi valori di $\rho$ (\si{\kilo\meter})\dots}