\chapter[Conclusioni]{Conclusioni}

Alla luce dei risultati ottenuti, il presente studio evidenzia il valore aggiunto del nuovo modello proposto nel campo della modellazione funzionale spazio-temporale come un'evoluzione rispetto al modello f-HDGM. Una delle distinzioni chiave di questo lavoro è l'introduzione del parametro di interazione spaziale $\rho$, il quale consente di affrontare situazioni e dinamiche in cui esiste interazione tra i punti di misura, un aspetto non considerato dal modello padre.
\par È importante notare che una delle limitazioni riguarda il processo di stima del nuovo parametro utilizzato nel modello proposto. Sebbene sia stata impiegata una metodologia affidabile come la cross-validazione per stimare il suo valore, è necessario sottolineare che tale approccio richiede notevoli risorse computazionali; questa necessità potrebbe costituire un ostacolo pratico per alcuni utenti, specialmente in contesti nei quali le risorse computazionali sono limitate. Per superare questo ostacolo, si rende necessaria la modifica dell'algoritmo EM affinché sia in grado di minimizzare anche la funzione $m(\rho)$, equazione~\ref{eq_f_rho}. Come anticipato nella sezione~\ref{metodologia}, la sua implementazione in D-STEM è già stata avviata, tuttavia richiede ancora un periodo di ricerca e sviluppo per poter rispondere ad alcuni quesiti ancora aperti. Innanzitutto, un aspetto cruciale risiede nella verifica dell'identificabilità di $\rho$, ossia capire se esso può essere stimato congiuntamente agli altri parametri $\boldsymbol{\theta}$, espressione~\ref{parametri_fp_HDGM}. In particolare, la stima di $\rho$ potrebbe andare in conflitto con quella di $\boldsymbol{\lambda}$ poiché entrambi vengono identificati tramite l'ottimizzazione numerica. Se ciò dovesse concretizzarsi, allora sarà opportuno stimare uno solo dei due parametri alla volta, tenendo fisso il restante. Dopodiché, un altro fattore da attenzionare riguarda l'inizializzazione di $\rho$. Essendo $m(\rho)$ una funzione non convessa, senza un'opportuna scelta del valore iniziale da assegnare al nuovo parametro si corre il rischio di identificare un minimo locale piuttosto che l'ottimo globale. Pertanto, per indirizzare l'ottimizzatore nella giusta direzione, è necessario avere un'idea a priori dell'ordine di grandezza di $\rho$, basandosi sulla conoscenza del dominio dello specifico caso di studio.
\par Infine, per quanto riguarda l'analisi svolta sul bike sharing, sarebbe interessante raffinare lo studio prendendo in esame un dataset pluriennale così da poter catturare anche la componente stagionale del fenomeno. Altresì, si potrebbero prendere in considerazioni delle variazioni metodologiche, per esempio eseguire un clustering dei punti di ritiro, a monte dell'analisi, ed eseguire così la stima di un modello spazio-temporale locale per ogni cluster. Ciò consentirebbe di rilassare un'importante assunzione fatta a priori, ossia che il parametro $\rho$ sia spazio-invariante.